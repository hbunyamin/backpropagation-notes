\documentclass[12pt]{article}

% fonts
\usepackage[scaled=0.92]{helvet}   % set Helvetica as the sans-serif font
\renewcommand{\rmdefault}{ptm}     % set Times as the default text font

% dmb: not mandatory, but i recommend you use mtpro for math fonts.
% there is a free version called mtprolite.

% \usepackage[amssymbols,subscriptcorrection,slantedGreek,nofontinfo]{mtpro2}

\usepackage[T1]{fontenc}
\usepackage{amsmath}
\usepackage{amsfonts}

% page numbers
\usepackage{fancyhdr}
\fancypagestyle{newstyle}{
\fancyhf{} % clear all header and footer fields
\fancyfoot[R]{\vspace{0.1in} \small \thepage}
\renewcommand{\headrulewidth}{0pt}
\renewcommand{\footrulewidth}{0pt}}
\pagestyle{newstyle}

% geometry of the page
\usepackage[top=1in, bottom=1in, left=1.625in, right=1.625in]{geometry}

% paragraph spacing
\setlength{\parindent}{0pt}
\setlength{\parskip}{2ex plus 0.4ex minus 0.2ex}

% useful packages
\usepackage{natbib}
\usepackage{epsfig}
\usepackage{url}
\usepackage{bm}


\usepackage{minted}
\usepackage{natbib}

\usepackage{tcolorbox}
\usepackage{etoolbox}
\BeforeBeginEnvironment{minted}{\begin{tcolorbox}}%
\AfterEndEnvironment{minted}{\end{tcolorbox}}%


\usepackage{amsmath,amsthm,amssymb,amsfonts}
\usepackage[shortlabels]{enumitem}

\usepackage{hyperref} % to create a URL link
\usepackage{tikz}

\newcommand{\N}{\mathbb{N}}
\newcommand{\Z}{\mathbb{Z}}

% ====================================================================
%   However, the environment can be renewed to support alignment:
% ====================================================================
\makeatletter
\renewcommand*\env@matrix[1][c]{\hskip -\arraycolsep
  \let\@ifnextchar\new@ifnextchar
  \array{*\c@MaxMatrixCols #1}}
\makeatother


\begin{document}

\begin{center}
  \Large \textbf{Penjelasan Algoritma Backpropagation} \\
  \vspace{0.1in}
  \normalsize Hendra Bunyamin \\
  \today
\end{center}

Lecture notes ini hendak memberikan penjelasan tambahan untuk video \href{https://www.coursera.org/lecture/machine-learning/backpropagation-algorithm-1z9WW}{\textit{Backpropagation Algorithm}} dari MOOC \href{https://www.coursera.org/learn/machine-learning/home/welcome}{\textit{Stanford Machine Learning}} di coursera.org. 

Penjelasan ini akan dimulai dengan \textit{logistic regression} (LR). \textit{Logistic regression} sebenarnya adalah \textit{neural networks} dengan 2 layer, yaitu \textit{input layer} dan \textit{output layer}. 

\section*{Logistic regression}
Contoh \textit{logistic regression} digambarkan pada Figure~\ref{fig:LR}. Model \textit{logistic regression} pada Figure~\ref{fig:LR} mempunyai 2 \textit{feature}, yaitu $x_1$ dan $x_2$. Ada juga $\theta_1$ (tanda panah yang menghubungkan $x_1$ dan $a^{(2)}_1$), dan $\theta_2$ (tanda panah yang menghubungkan $x_2$ dan $a^{(2)}_1$). Terdapat juga \textit{bias term} ($x_0$) yang tidak tergambar secara eksplisit dan $\theta_0$ (tanda panah yang menghubungkan $x_0$ dan $a^{(2)}_1$).

\def\layersep{2.5cm}
\begin{figure}[!ht]
	\centering
\begin{tikzpicture}[shorten >=1pt,->,draw=black!50, node distance=\layersep]
    \tikzstyle{every pin edge}=[<-,shorten <=1pt]
    \tikzstyle{neuron}=[circle,fill=black!25,minimum size=17pt,inner sep=0pt]
    \tikzstyle{input neuron}=[neuron, fill=green!50];
    \tikzstyle{output neuron}=[neuron, fill=red!50];
    \tikzstyle{hidden neuron}=[neuron, fill=blue!50];
    \tikzstyle{annot} = [text width=4em, text centered]

    % Draw the input layer nodes
    \foreach \name / \y in {1,...,2}
    % This is the same as writing \foreach \name / \y in {1/1,2/2,3/3,4/4}
        \node[input neuron, pin=left:$x_{\y}$] (I-\name) at (0,-\y) {};

    % Draw the output layer node
    \node[output neuron,pin={[pin edge={->}]right:$h_{\theta}$}, right of=I-1] (O) {};
	
    % Connect every node in the input layer with the node in the
    % output layer.
    \foreach \source in {1,...,2}
		\path (I-\source) edge (O);		 	
\end{tikzpicture}	
\caption{\textit{Logistic regression} dengan 2 \textit{feature}, yaitu $x_1$ dan $x_2$}
\label{fig:LR}
\end{figure}

Model \textit{logistic regression} ($h_\theta$) memiliki bentuk
\begin{equation}
	h_\theta(x_1, x_2) = \frac{1}{1+e^{-(\theta_0 + \theta_1 x_1 + \theta_2 x_2)}}
\end{equation}

atau bentuk \textit{vectorized}-nya adalah

\begin{equation}
	h_\theta(x) = \frac{1}{1+e^{-\theta^T x}}
\end{equation}

dengan $\theta = \begin{bmatrix}[r]
	\theta_0 & \theta_1 & \theta_2
\end{bmatrix}^T$ dan $x = \begin{bmatrix}[r]
	1 & x_1 & x_2
\end{bmatrix}^T$.

\subsection*{Gradient dari Logistic Regression untuk satu training instance}
Misalkan terdapat satu \textit{training instance}, yaitu $x = \begin{bmatrix}[r]
	1 & x_1 & x_2
\end{bmatrix}^T$ dan $y$. Akan dihitung \textit{gradient} dari model \textit{logistic regression} tersebut.

\begin{align}
	\frac{\partial J(\theta)}{\partial \theta_0} &= (h_\theta(x) - y), \label{eq:gradient-lr-1} \\
	\frac{\partial J(\theta)}{\partial \theta_1} &= (h_\theta(x) - y) x_1, \text{ dan} \label{eq:gradient-lr-2} \\	
	\frac{\partial J(\theta)}{\partial \theta_2} &= (h_\theta(x) - y) x_2 \label{eq:gradient-lr-3}	
\end{align}
Yang perlu dicatat adalah bahwa untuk \textit{logistic regression} dengan satu \textit{training instance},
\begin{equation*}
	\delta^{(2)}_1 = (h_\theta(x) - y).
\end{equation*}
$\delta^{(2)}_1$ ini dibaca \textit{delta} dan merupakan \textit{selisih prediksi model dengan nilai sebenarnya}.
 
Dengan menggunakan notasi $\delta^{(2)}_1$, Persamaan~(\ref{eq:gradient-lr-1}), (\ref{eq:gradient-lr-2}), dan (\ref{eq:gradient-lr-3}) dapat ditulis menjadi
\begin{align}
	\frac{\partial J(\theta)}{\partial \theta_0} &= \delta^{(2)}_1, \label{eq:gradient-lr-delta-1} \\
	\frac{\partial J(\theta)}{\partial \theta_1} &= \delta^{(2)}_1 x_1, \text{ dan} \label{eq:gradient-lr-delta-2} \\	
	\frac{\partial J(\theta)}{\partial \theta_2} &= \delta^{(2)}_1 x_2 \label{eq:gradient-lr-delta-3}	
\end{align}
Yang perlu diperhatikan adalah bahwa setiap \textit{neuron} selain \textit{neuron}-\textit{neuron} di \textit{input layer} memiliki delta ($\delta$). Lebih lanjut,
\begin{equation*}
	\text{gradient di garis atau bobot} =  \delta \times \text{input di garis atau bobot}.
\end{equation*}

\subsection*{Gradient dari LR untuk banyak training instance}
Diketahui $m$ \textit{training instances} dan \textit{label}nya, yaitu 
\begin{equation*}
X = \begin{bmatrix}[c]
	1       & x^{(1)}_1 & x^{(1)}_2 \\
	1       & x^{(2)}_1 & x^{(2)}_2 \\	
	\vdots  & \vdots    & \vdots \\
	1       & x^{(m)}_1 & x^{(m)}_2 \\		 
\end{bmatrix} \text{ dan } y = \begin{bmatrix}[r]
	y^{(1)} & y^{(1)} & \cdots & y^{(m)}
\end{bmatrix}^T	
\end{equation*} 
\textit{Gradient} dari model \textit{logistic regression} dengan banyak \textit{training instance} ini adalah
\begin{align}
	\frac{\partial J(\theta)}{\partial \theta_0} &= \sum_{i=1}^{m}{(h_\theta(x) - y)}, \label{eq:gradient-lr-multi-1} \\
	\frac{\partial J(\theta)}{\partial \theta_1} &= \sum_{i=1}^{m}{(h_\theta(x) - y) x_1}, \text{ dan} \label{eq:gradient-lr-multi-2} \\	
	\frac{\partial J(\theta)}{\partial \theta_2} &= \sum_{i=1}^{m}{(h_\theta(x) - y) x_2} \label{eq:gradient-lr-multi-3}	
\end{align}

Dengan pemahaman ini, selanjutnya kita akan menghitung gradient dari \textit{neural network}.

\begin{figure}[!ht]
	\centering
\begin{tikzpicture}[shorten >=1pt,->,draw=black!50, node distance=\layersep]
    \tikzstyle{every pin edge}=[<-,shorten <=1pt]
    \tikzstyle{neuron}=[circle,fill=black!25,minimum size=17pt,inner sep=0pt]
    \tikzstyle{input neuron}=[neuron, fill=green!50];
    \tikzstyle{output neuron}=[neuron, fill=red!50];
    \tikzstyle{hidden neuron}=[neuron, fill=blue!50];
    \tikzstyle{annot} = [text width=4em, text centered]

    % Draw the input layer nodes
    \foreach \name / \y in {1,...,2}
    % This is the same as writing \foreach \name / \y in {1/1,2/2,3/3,4/4}
        \node[input neuron, pin=left:$x_{\y}$] (I-\name) at (0,-\y) {};

    % Draw the hidden layer nodes
    \foreach \name / \y in {1,...,2}
        \path[yshift=0.1cm]
            node[hidden neuron] (H-\name) at (\layersep,-\y cm) {};

    \foreach \name / \y in {1,...,2}
        \path[yshift=0.1cm]
            node[hidden neuron] (J-\name) at (2*\layersep,-\y cm) {};


    % Draw the output layer node
    \node[output neuron,pin={[pin edge={->}]right:$(h_{\Theta})_1$}, right of=J-1] (O) {};
    \node[output neuron,pin={[pin edge={->}]right:$(h_{\Theta})_2$}, right of=J-2] (1) {};
	
    % Connect every node in the input layer with every node in the
    % hidden layer.
    \foreach \source in {1,...,2}
        \foreach \dest in {1,...,2}
            \path (I-\source) edge (H-\dest);
            
    \foreach \source in {1,...,2}
        \foreach \dest in {1,...,2}
            \path (H-\source) edge (J-\dest);
            
    % Connect every node in the hidden layer with the output layer
    \foreach \source in {1,...,2}{
        \path (J-\source) edge (O);
        \path (J-\source) edge (1);
	}
	
    % Annotate the layers
    \node[annot,above of=H-1, node distance=1cm] (hl) {Layer 2};
    \node[annot,above of=O, node distance=1cm] (ol) {Layer 4};    
    \node[annot,left of=hl] {Layer 1};
    \node[annot,right of=hl] {Layer 3};
\end{tikzpicture}	
\caption{\textit{Artificial neural network} untuk \textbf{Problem 1}}
\label{fig:ANN-soal-1}
\end{figure}


$\Theta_{11}$


What if the knowledge and data we have are not sufficient to completely determine the correct classifier? Then we run the risk of just hallucinating a classifier (or parts of it) that is not grounded in reality, and is simply encoding random quirks in the data. \textit{This problem is called overfitting}, and is \textit{the bugbear of machine learning}.
When your learner outputs a classifier that is 100\% accurate on the training data but only 50\% accurate on test data, when in fact it could have output one that is 75\% accurate on both, it has overfit.

Everyone in machine learning knows about \textit{overfitting}, but it comes in many forms that are not immediately obvious. One way to understand overfitting is by \textit{decomposing generalization error into bias and variance}.  \textit{Bias is a learner's tendency to consistently learn the same wrong thing}. \textit{Variance is the tendency to learn random things irrespective of the real signal}. Figure~\ref{fig:darts} illustrates this by an analogy with throwing darts at a board. A linear learner has high bias, because when the frontier between two classes is not a hyperplane the learner is unable to induce it~\citep{domingos2012few}.

\begin{figure}[!ht]
	\centering
	\includegraphics[scale=.25]{images/darts}
	\caption{Bias and variance in dart-throwing~\citep{domingos2012few}.}
	\label{fig:darts}
\end{figure}

What do we mean by the variance and bias of a statistical learning method? Variance refers to the amount by which $h_\theta$ would change if we estimated it using a different training data set. Since the training data
are used to fit the statistical learning method, different training data sets will result in a different $h_\theta$. But ideally the estimate for $y$ should not vary too much between training sets. However, if a method has high variance
then small changes in the training data can result in large changes in $h_\theta$~\citep{james2013introduction}.

On the other hand, bias refers to the error that is introduced by approximating a real-life problem, which may be extremely complicated, by a much simpler model. For example, linear regression assumes that there is a linear
relationship between $Y$ and $x_1$, $x_2$, $\ldots$, $x_n$. It is unlikely that any real-life problem truly has such a simple linear relationship, and so performing linear regression will undoubtedly result in some bias in the estimate of $y$.


\bibliographystyle{apalike}
\bibliography{references}

\end{document}
